\begin{abstract}

The Exascale Computing Project (ECP) Software Technology (ST) Focus Area is responsible for developing critical software capabilities that will enable successful execution of ECP applications, and for providing key components of a productive and sustainable Exascale computing ecosystem that will position the US Department of Energy (DOE) and the broader high performance (HPC) community with a firm foundation for future extreme-scale computing capabilities.

This \textit{ECP ST Capability Assessment Report (CAR)} provides an overview and assessment of current ECP ST capabilities and activities, giving stakeholders and the broader HPC community information that can be used to assess ECP ST progress and plan their own efforts accordingly.  ECP ST leaders commit to updating this document on regular basis (targeting approximately every six months). Highlights from the report are presented here.  

\textbf{What is new in CAR V1.5:} CAR V1.5 contains the following updates relative to CAR V1.0.
\begin{itemize}
	\item The two-page summaries of each ECP L4 projects have been updated to reflect recent progress and next steps. See Section~\ref{sect:project-summaries}.
	\item The Extreme-scale Scientific Software Stack (E4S) is introduced.  The first release was November 8, 2018. The second release was January 2019. E4S is the primary integration and delivery vehicle for ECP ST capabilities.  See Section~\ref{subsubsect:e4s}.
	\item The ECP ST SDK effort has defined its initial grouping into 6 product suites.  See Section~\ref{subsubsect:sdks}.
	\ifpublic
	\else
	\item ECP ST has embarked on planning and preparations for FY2020 and the anticipated transition to Critical Decision Phase 2.  See Sections A and B in the Appendix.
	\fi
\end{itemize}

\textbf{The Exascale Computing Project Software Technology (ECP ST) focus area represents the key bridge between Exascale systems and the scientists developing applications that will run on those platforms:} ECP ST efforts contribute to 89 software products (Section~\ref{subsect:products}) in five technical areas (Table~\ref{table:wbs}). 33 of the 89 products are broadly used in the HPC community and require substantial investment and transformation in preparation for Exascale architectures.  An additional 23 are important to some existing applications and typically represent new capabilities that enable new usage models for realizing the potential that Exascale platforms promise.  The remaining products are in early development phases, addressing emerging challenges and opportunities that Exascale platforms present.

\textbf{\pmr:} ECP ST is developing key enhancements to MPI and OpenMP, addressing in particular the important design and implementation challenges of combining massive inter-node and intra-node concurrency into an application.  We are also developing a diverse collection of products that further address next generation node architectures, to improve realized performance, ease of expression and performance portability.

\textbf{\tools:}  We are enhancing existing widely used compilers (LLVM) and performance tools for next-generation platforms. Compilers are critical for heterogeneous architectures, and LLVM is the most popular compiler for heterogeneous systems. 
%
As node architectures become more complicated and concurrency even more necessary, compilers must generate optimized code for many architectures, and the impediments to performance and scalability become even harder to diagnose and fix.  
%
Development tools provide essential insight into these performance challenges and code transformation and support capabilities that help software teams generate efficient code, utilize new memory systems and more.

\textbf{\mathlibs:} High-performance scalable math libraries have enabled parallel execution of many applications for decades.  ECP ST is providing the next generation of these libraries to address needs for latency hiding, improved vectorization, threading and strong scaling.  In addition, we are addressing new demands for system-wide scalability including improved support for coupled systems and ensemble calculations.  The math libraries teams are also spearheading the software development kit (SDK) initiative that is a pillar of the ECP ST software delivery strategy (Section~\ref{subsubsect:sdks}).

\textbf{\dataviz:} ECP ST has a large collection of data management and visualization products that provide essential capabilities for compressing, analyzing, moving and managing data.  These tools are becoming even more important as the volume of simulation data we produce grows faster than our ability to capture and interpret it.

\textbf{\ecosystem:} This new technical area of ECP ST provides important enabling technologies such as containers and experimental OS environments that allow ECP ST to provide requirements, analysis and design input for vendor products.  This area also provides the critical resources and staffing that will enable ECP ST to perform continuous integration testing, and product releases.  Finally, this area engages with software and system vendors, and DOE facilities staff to assure coordinated planning and support of ECP ST products.

 \textbf{ECP ST Software Delivery mechanisms:} ECP ST delivers software capabilities to users via several mechanisms (Section~\ref{sect:deliverables}).  Almost all products are delivered via source codes to at least some of their users.  Each of the major DOE computing facilities provides direct support of some users for about 20 ECP ST products.  About 10 products are available via vendor software stack and via binary distributions such as Linux distributions.
 
 \textbf{ECP ST Project Restructuring:} ECP ST completed a significant restructuring in November  2017 (Section~\ref{subsect:ProjectRestructuring}).  We reduced the number of technical areas from 8 to 5 and reduced the number of L4 projects significantly by simplifying the organization of ATDM projects.  We introduced new projects for software development kits that are a key organizational feature for designing, testing and delivering our software.  Finally, we introduced a new technical area (\ecosystem) that provides the critical capabilities we need for delivering a sustainable software ecosystem.  ECP ST is in the process of further restructuring to prepare for the CD-2 phase of the project.  Details will be provided in the CAR V2.0 in July 2019.
 
 \textbf{ECP ST Project Overviews:} A significant portion of this report includes 2-page synopses of each ECP ST project (Section~\ref{sect:project-summaries}), including a project overview, key challenges, solution strategy, recent progress and next steps.
 	
 \textbf{Project organization:} ECP ST has established a tailored project management structure using impact goals/metrics, milestones, regular project-wide video meetings, monthly and quarterly reporting, and an annual review process.  This structure supports project-wide communication, and coordinated planning and development that enables 55 projects and more than 250 contributors to create the ECP ST software stack.

\end{abstract}

\subsection{\pmr}

\textbf{End State:} A cross-platform, production-ready programming environment that enables and accelerates the development of mission-critical software at both the node and full-system levels.

\subsubsection{Scope and Requirements}
A programming model provides the abstract design upon which developers express and coordinate the efficient parallel execution of their program. A particular model is implemented as a developer-facing interface and a supporting set of runtime layers. To successfully address the challenges of exascale computing, these software capabilities must address the challenges of programming at both the node- and full-system levels. These two targets must be coupled to support multiple complexities expected with exascale systems (e.g., locality for deep memory hierarchies, affinity for threads of execution, load balancing) and also provide a set of mechanisms for performance portability across the range of potential and final system designs. Additionally, there must be mechanisms for the interoperability and composition of multiple implementations (e.g., one at the system level and one at the node level). This must include abilities such as resource sharing for workloads that include coupled applications, supporting libraries and frameworks, and capabilities such as in situ analysis and visualization. 

Given the ECP’s timeline, the development of new programming languages and their supporting infrastructure is infeasible. We do, however, recognize that the augmentation or extension of the features of existing and widely used languages (e.g., C/C++ and Fortran) could provide solutions for simplifying certain software development activities. 

\subsubsection{Assumptions and Feasibility}
The intent of the PMR L3 is to provide a set of programming abstractions and their supporting implementations that allow programmers to select from options that meet demands for expressiveness, performance, productivity, compatibility, and portability. It is important to note that, while these goals are obviously desirable, they must be balanced with an additional awareness that today’s methods and techniques may require changes in both the application and the overall programming environment and within the supporting software stack.

\subsubsection{Objectives}
PMR provides the software infrastructure necessary to enable and accelerate the development of HPC applications that perform well and are correct and robust, while reducing the cost both for initial development and ongoing porting and maintenance. PMR activities need to reflect the requirements of increasingly complex application scenarios, usage models, and workflows, while at the same time addressing the hardware challenges of increased levels of concurrency, data locality, power, and resilience. The software environment will support programming at multiple levels of abstraction that includes both mainstream as well as alternative approaches if feasible in ECP’s timeframe. 

Both of these approaches must provide a portability path such that a single application code can run well on multiple types of systems, or multiple generations of systems, with minimal changes. The layers of the system and programming environment implementation will therefore aim to hide the differences through compilers, runtime systems, messaging standards, shared-memory standards, and programming abstractions designed to help developers map algorithms onto the underlying hardware and schedule data motion and computation with increased automation.
\subsubsection{Plan}
PMR contains fifteen L4 projects. To ensure relevance to DOE missions, these efforts leverage and collaborate with existing activities within the broader HPC community. Initial efforts focus on identifying the core capabilities needed by the selected ECP applications and components of the software stack, identifying shortcomings of current approaches, establishing performance baselines of existing implementations on available petascale and prototype systems, and the re-implementation of the lower-level capabilities of relevant libraries and frameworks. These efforts provide demonstrations of parallel performance of algorithms on pre-exascale, leadership-class machines--at first on test problems, but eventually in actual applications (which will require close collaboration with the AD and HI teams). Initial efforts also inform research into exascale-specific algorithms and requirements that will be implemented across the software stack. The supported projects target and implement early versions of their software on CORAL, NERSC and ACES pre-exascale systems--with an ultimate target of production-ready deployment on the exascale systems. Throughout the effort, the applications teams and other elements of the software stack evaluate and provide feedback on their functionality, performance, and robustness. Progress towards these goals is documented quarterly and evaluated annually (or more frequently if needed) based on PMR-centric milestones as well as joint milestone activities shared across associated software stack activities by Application Development and Hardware \& Integration focus areas.

\subsubsection{Risks and Mitigation Strategies}
The mainstream activities of PMR focus on advancing the capabilities of the Message Passing Interface (MPI) and OpenMP. Pushing them as far as possible into the exascale era is key to supporting an evolutionary path for applications. This is the primary risk mitigation approach for both PMR and existing application codes. Extensions to MPI and OpenMP standards will require research, and part of the efforts will focus on rolling these findings into existing standards, which takes time. To further address risks, PMR is exploring alternative approaches to mitigate the impact of potential limitations of the MPI and OpenMP programming models. This effort is tracked using the risk register.

Another risk is the failure of adoption of the software stack by the vendors, which is tracked in the risk register, and mitigated by the specific delivery focus in sub-element SW Ecosystems and Delivery. Past experience has shown that a combination of laboratory-supported open source software and vendor-optimized solutions built around standard APIs that encourage innovation across multiple platforms is a viable approach and what we are doing in PMR. We are using close interaction with the vendors early on to encourage adoption of the software stack, including well-tested practices of including support for key software products or APIs into large procurements through NRE or other contractual obligations. A mitigation strategy for this approach involves building a long-lasting open source community around projects that are supported via laboratory and university funding. This approach is being extended to other APIs and alternative models (that are being defined and eventually standardized) to allow for deeper and stack-wide introspection as well as resource sharing.

Creating a coordinated set of software requires strong management to ensure that duplication of effort is minimized. This is recognized by ECP management, and processes are in place to ensure collaboration is effective, shortcuts are avoided unless necessary, and an agile approach to development is instituted to prevent prototypes moving directly to product. The duplication of effort specifically, and the overall integration of the software stack, are tracked in the risk register. 

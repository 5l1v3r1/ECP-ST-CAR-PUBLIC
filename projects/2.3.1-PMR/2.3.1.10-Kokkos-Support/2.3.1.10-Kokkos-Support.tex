\subsubsection{\stid{1.10} Kokkos Support} 

\paragraph{Overview} 
Kokkos Support provides documentation, training and community building support for the Kokkos programming model. 
To that end, the project develops programming guides, API references and tutorial material for Kokkos which is presented both as independent events and at conferences such as Supercomputing.
Kokkos Support is also responsible for setting up community interaction channels such as the Slack channels now used for user communication and fostering the GitHub interactions between Kokkos developers and users.
Finally, the project supports engagement with the C++ standard in order to further the adoption of successful Kokkos concepts into the core language. 

\paragraph{Key Challenges}
For a new programming model to be successful, a comprehensive support and training infrastructure is absolutely critical. 
Prospective users must learn how to use the programming model, current users must be able to bring up issues with the development team and access detailed documentation, and the development team of the model must be able to continue technical efforts without being completely saturated with support tasks. 
The latter point became a significant concern for the Kokkos team with the expected growth of the user base through ECP.  
Already before the launch of ECP, there were multiple application or library teams starting to use Kokkos for each developer on the core team -- a level not sustainable into the future without a more scalable support infrastructure. 
This issue was compounded by the fact that Kokkos development was funded through NNSA projects, making it hard to justify extensive support for open science applications. 

\paragraph{Solution Strategy}

Kokkos Support addresses these challenges through a number of ways. 
First and foremost, it provides explicit means for supporting all DOE ECP applications. 
A main component of that is funding for local Kokkos experts at the Sandia, Oak Ridge and Los Alamos laboratories which can serve as direct contacts for local applications and, in Oak Ridge's case, for users of the Oak Ridge Leadership Computing Facility. 
Secondly, the project develops a reusable support infrastructure, which makes supporting more users scalable and cost effective. 

The support infrastructure consists of GitHub wiki pages for the programming guide and an API reference, GitHub issues to track feature requests and bug reports, a Slack channel for direct user-to-user and user-to-developer communication, and tutorial presentations and cloud-based Kokkos hands-on exercises. 

\paragraph{Recent Progress}

Kokkos Support has successfully run multiple Kokkos bootcamps for DOE applications as well as organized a number of single day tutorials. 
At the most recent tutorial during the DOE ECP All Hands meeting, the new cloud-based hands-on infrastructure was used for the first time, allowing tutorial attendees to use GPUs on remote servers without the hassle of temporary user account administration at DOE computing facilities. 
The project also improved existing documentation and transferred it to GitHub wiki pages which are tailored for software documentation and more maintainable. 
The Slack channels usage is growing, which have seen participation from users across the DOE and Kokkos community. 
There are also more interactions on GitHub issues, including a number of pull requests volunteered from external users as a result of these interactions to improve small parts of the Kokkos implementation. 
These latter two points are a sign that the community is growing and more actively participating in advancing Kokkos, a necessary step for a more sustainable future where users may answer other users questions, and the community begins to provide new features and solutions to problems. 
 
\paragraph{Next Steps}

There are two main thrusts of development underway: the writing of a Kokkos API Reference and the development of more advanced tutorials. 
While the Kokkos Programming Guide and the Tutorial Presentations are well received, more advanced users often only want to look up the API of Kokkos features. 
Such an API reference is not yet available. Its development is under way, and we hope to have it cover the majority of Kokkos features by the end of Summer 2018. 

For tutorials, feedback provided by attendees indicate that some of the material covered in the standard tutorial is a bit too advanced for an introduction to Kokkos. On the other hand, as the number of users who have had previous exposure to Kokkos is growing, they are asking for more of the advanced features to be covered. To support both types of attendees, it is clear that splitting the tutorial into beginner and advanced sections, as well as extending the advanced section beyond what is currently covered is necessary. 

Finally, we would like to improve the accessibility of all of the resources that are being developed to support Kokkos and increase its adoption. 
Currently, all documentation, tutorials, references, and support channels are in various locations that are best suited to their differing format requirements. 
However, having a one-stop landing page where new encounters can learn about the project and current users can find the location of all available resources would increase usage of the various materials and communication channels within the community, among both developers and users.


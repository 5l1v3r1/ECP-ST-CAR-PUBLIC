\subsubsection{\stid{1.06} ISC4MCM (RAJA)} 

\paragraph{Overview.} 
The Integrated Software Components for Managing Computation and Memory 
Interplay at Exascale (ISC4MCM) project is providing software libraries that 
enable application and library developers to meet advanced architecture 
portability challenges. The project goals are to enable writing performance 
portable computational kernels and coordinate complex heterogeneous memory 
resources among components in a large integrated application. These 
libraries enhance developer productivity by insulating them from much of the 
complexity associated with parallel programming model usage and 
system-specific memory concerns.

The software products provided by this project are three complementary and 
interoperable libraries:
\begin{enumerate}
\item {\bf RAJA:} Software abstractions that enable C++ developers to write
  performance portable (i.e., single-source) numerical kernels (loops). 
\item {\bf CHAI:} C++ ``managed array'' abstractions that enable transparent
  and automatic copying of application data to execution memory spaces at run
    time as needed based on RAJA execution contexts.
\item {\bf Umpire:} A portable memory resource management library that provides
  a unified high-level API for resource discovery, memory provisioning,
    allocation, access, operations, and introspection.
\end{enumerate}

Capabilities delivered by these software efforts are needed to manage the
diversity and uncertainty associated with current and future HPC architecture
design and software support. Moving forward, ECP applications and libraries 
need to achieve performance portability: without becoming bound to particular
(potentially-limiting) hardware or software technologies, by insulating 
numerical algorithms from platform-specific data and execution concerns, and 
without major disruption as new machine, programming models, and vendor
software become available.

These libraries in development in this project are currently used in production
ASC applications at Lawrence Livermore National Laboratory (LLNL). They are
also being used or being explored/adopted by several ECP application and
library projects, including: LLNL ATDM application, GEOS (Subsurface), SW4
(EQSIM), MFEM (CEED co-design), and SUNDIALS.

The software projects are highly-leveraged with other efforts. Team members
include: ASC and ATDM application developers, ASD tool developers, university
collaborators, and vendors. This ECP ST project supports outreach to the ECP
community and collaboration with ECP efforts.

\paragraph{Key Challenges.}

The main technical challenge for this project is enabling production
applications to achieve performance portability in an environment of rapidly
changing, disruptive HPC hardware architecture design. Typical large
applications contain $O(10^5) - O(10^6)$ lines of code and $O(10K)$ loop
kernels. The codes must run efficiently on platforms ranging from laptops to
commodity clusters to large HPC platforms. The codes are long-lived and are
used daily for decades, so they must be portable across machine generations.
Also, the codes are under continual development, with a steady stream of new
capabilities added throughout their lifetimes -- continual validation and
verification is essential, which precludes substantial rewrites from scratch.
Lastly, the complex interplay of multiple physics packages and dozens of
libraries makes it so that the data required for the full set of components
needed for a given simulation may not fit into a single system memory space. To
advance scientific computing capabilities, applications must navigate these
constraints while facing substantial hardware architecture disruption along the
road toward Exascale computing platforms. 

While the software provided by this project has a substantial user base at
LLNL, achieving broader adoption in the ECP (projects without LLNL involvement,
in particular) is another challenge. The software efforts are funded almost
entirely by LLNL programs and the majority of their developers work on LLNL
application projects. So resource limitations is a key issue.

\paragraph{Solution Strategy.}

The software libraries in this project focus on encapsulation and 
application-facing APIs to insulate users from the complexity and 
challenges associated with diverse forms of parallelism and heterogeneous 
memory systems. This approach allows users to exploit new capabilities 
with manageable rewriting of their applications.

RAJA provides various C++ abstractions for parallel loop execution. It
supports: various parallel programming model back-ends, such as OpenMP 
(CPU multithreading and target offload), CUDA, Intel Threading Building Blocks,
etc.; loop iteration space and data view constructs to reorder, 
aggregate, tile, and partition loop iterations; complex loop kernel 
transformations for optimization, such as reordering loop nests, fusing 
loops, etc. RAJA also supports portable atomic operations, parallel scans, 
and CPU and GPU shared memory. After loops have been converted to RAJA, 
developers can explore implementation alternatives via RAJA features without 
altering loop kernels at the application level.

CHAI provides C++ ``managed array'' abstractions that automatically copy 
data to execution memory spaces as needed at run time based on RAJA execution 
contexts. Access to array data in loop kernels looks the same as when using
traditional C-style arrays.

Umpire provides a portable API for managing complex memory resources by 
providing uniform access to other libraries and utilities that provide
system-specific capabilities. Umpire decouples resource allocation from 
specific memory spaces, allocators, and operations. The memory introspection 
functionality of Umpire enables applications and libraries to make memory 
usage decisions based on allocation properties (size, location, sharing 
between packages, etc.)

All three software libraries are open source and available on
GitHub~\cite{RAJA-github, CHAI-github, Umpire-github}. There they provide
regular software and documentation releases. Each project has dedicated email
lists, issue tracking, test suites, and automated testing.

\paragraph{Recent Progress}

In FY18, CHAI and Umpire have been released as open source software projects
and they are now developed on GitHub Recent development has focused on 
user documentation and cleaner integration of these two libraries to give 
applications more flexible and easy access to their capabilities.

Many new features have been added to RAJA in FY18 to enable flexible
loop transformations for complex loop kernels via execution policies.
LLNL applications are assessing this new functionality now in a 
"pre-release" version; it will be generally available before the end of FY18.

The RAJA Performance Suite~\cite{RAJAPerf-github} was released and made 
available on Github in January 2018. The Suite is used to assess and track 
performance of RAJA across programming models and diverse loop 
kernels. It is also being used for compiler acceptance testing in the CORAL 
procurement and was prepared for use as a benchmark for the CORAL-2 procurement.

In 2018, the RAJA project expanded its visibility beyond DOE NNSA Labs. 
Recent presentations include a RAJA tutorial at the 2018 ECP Annual Meeting 
and an application use case study the 2018 NVIDIA GPU Tech Conference (GTC). 
Future tutorials are planned at 2018 ATPESC and GTC 2019. Also, a RAJA paper 
and $1/2$-day tutorial proposal were submitted to SC18.

\paragraph{Next Steps}

Our next efforts include:
\begin{enumerate}
\item {\bf Fill RAJA Gaps:} Not all features are available for all programming
  model back-ends; as models mature, such as OpenMP4.5, these gaps will be
    filled.
\item {\bf Expand RAJA User Guide and Tutorial:} Build example codes and user
  documentation for latest RAJA features and prepare for future tutorials
    (ATPESC 2018 and SC18).
\item {\bf Expand RAJA Performance Suite:} Include kernels that exercise more
  application use cases and RAJA features.
\item {\bf Focus RAJA Vendor Interaction:} Work with CORAL vendors to address
  issues as applications port to the Sierra platform at LLNL; establish early
    interactions with CORAL-2 vendors to ensure RAJA will be supported well on
    CORAL-2 systems.
\item {\bf Expand Umpire Capabilities:} Explore potential collaboration with
  relevant ECP efforts, such as SICM project.
\end{enumerate}

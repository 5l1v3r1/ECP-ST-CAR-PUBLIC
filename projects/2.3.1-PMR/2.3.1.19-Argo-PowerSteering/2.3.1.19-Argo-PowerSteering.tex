\subsubsection{\stid{1.19} Argo: Low-Level Resource Management for the OS and Runtime}

The goal of the Argo project~\cite{perarnau2017argo} is to augment and
optimize existing OS/R components for use in production HPC systems,
providing portable, open source, integrated software that improves the
performance and scalability of and that offers increased functionality to
exascale applications and runtime systems.

System resources should be managed in cooperation with applications and
runtime systems to provide improved performance and resilience. This is
motivated by the increasing complexity of HPC hardware and application
software, which needs to be matched by corresponding increases in the
capabilities of system management solutions.

The Argo software is developed as a toolbox---a collection of autonomous
components that can be freely mixed and matched to best meet the user's
needs.

Project activities focus around four products:
\begin{itemize}
\item AML --- a library providing explicit, application-aware memory
management for deep memory systems,

\item UMap --- a user level library incorporating NVRAM into complex memory
hierarchy using a high performance \texttt{mmap}-like interface.

\item PowerStack --- power management infrastructure optimizing performance
of Exascale applications under power or energy constraints.

\item NRM --- a daemon centralizing node management activities such as
job management, resource management, and power management.
\end{itemize}

\input projects/2.3.1-PMR/2.3.1.19-Argo-PowerSteering/aml
\input projects/2.3.1-PMR/2.3.1.19-Argo-PowerSteering/umap
\input projects/2.3.1-PMR/2.3.1.19-Argo-PowerSteering/powerstack
\input projects/2.3.1-PMR/2.3.1.19-Argo-PowerSteering/nrm

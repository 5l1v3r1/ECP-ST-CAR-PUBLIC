\subsection{\stid{6} \nnsa}\label{subsect:nnsa}

\textbf{End State:} Software used by the NNSA/ATDM National Security
Applications and associated exascale facilities, hardened for
production use and made available to the larger ECP community

\subsubsection{Scope and Requirements}
The NNSA ST L3 area is new in FY20, although the projects included
have all been part of the ECP in the past. The capabilities of these
software products remains aligned with the other Software Technology
L3 areas from which they were derived, but are managed separately for
non-technical reasons out of scope for this document.

The resulting products in this L3 area 
are open source, important or critical to the success of the NNSA
National Security Applications, and are used (or potentially used) in
the broader ECP community. The products in this L3 span the scope of
the rest of ST (Programming Models and Runtimes, Development Tools,
Math Libraries, Data Analysis and Vis, and Software Ecosystem), and
will be coordinated with those other L3 technical 
areas through a combination of existing relationships and
cross-cutting efforts such as the ST SDKs and E4S.  

\subsubsection{Objectives}

The objective of these software products are to support the
development of new from-scratch applications within the NNSA that were
started just prior to the founding of the ECP under the ATDM (Advanced
Technology Development and Mitigation) program element within NNSA and
ASC. While earlier incarnations of these products may have been more
research-focused, by the time of the ECP ST restructuring in 2019 that
resulted in this L3 area, these products are in regular use by their
ATDM applications, and have matured to the point where they are ready
for use within the broader open source community.

\subsubsection{Plan}

NNSA ST products are developed along with and alongside a broader
portfolio of ASC products in an integrated program, and are planned
out at high level in the annual ASC Implementation Plan, and in detail
using approved processes within the home institution/laboratory. They
are scoped to have 
resources sufficient for the success of the NNSA mission, as well as a
modicum of community support (e.g. maintaining on GitHub, or answering
occasional questions from the community).

For ECP products not part of the NNSA portfolio that have critical
dependencies on these products, there are often other projects within
ECP that provide additional funding and scope for those activities. In
those cases, there may be additional information within this document
on these products.


\subsubsection{Risks and Mitigations Strategies}

A primary risk within this L3 area is that the 2020 ASC Level 1 milestone
designed as a capstone for the ATDM initiative and a decision point
for the ultimate transition of those applications into the core ASC
porfolio, will fail due to the in adequacy of these software
products. While not all of them are on the critical path to
application success (instead focusing on productivity enhancements for
end users, or analysis functionality), it is expected that first and
foremost they will contribute to the success of that milestone, as any
subsequent ASC milestones and decision points about the ultimate fate
of those applications. Mitigation is to use other ASC funding to bolster these
efforts as needed.

A secondary risk is that others in the community will pick up these
products as open source, and expect additional support beyond the
scope of the primary NNSA. If those dependant products are within the
ECP, the main mitigation is to use ASCR contingency funding to provide
additional development and support - potentially through support of teams
outside of the home institution. If those dependant products are in
the broader community, mitigations are generally outside of the scope
of the ECP - although each NNSA lab typically has some sort of project
(or possibly even a policy) on how to deal with external demands on
open source products.




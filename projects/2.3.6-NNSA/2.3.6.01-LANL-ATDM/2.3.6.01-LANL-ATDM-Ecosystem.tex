\subsubsection{\stid{6.01} LANL ATDM Software Ecosystem \& Delivery Projects - Resilience Subproject} 

\paragraph{Overview}
The resilience subproject has a goal of providing tools and solutions for
users and administrators of the Exascale machines to deal with expected
unreliable operations.  Due to the extreme scale and bleeding-edge components
necessary to meet the needs of the ECP project, it is expected that the
applications will need to deal with application / job interrupts at a rate
higher than users are used to on conventional extreme-scale systems of today.
This also means that datacenter operations will be affected and modeling tools
needs to be provided to monitor and be prepared for this eventuality.

In general, this subproject approaches this through a combination of data
collection of today's extreme-scale systems (monitoring), data analytics,
machine learning, model building, and fault injection into applications at
scale.

\paragraph{Key Challenges}
The key challenges in this area are that it is difficult to acquire data on
contemporary systems for a number of reasons such as NDA, security, and 
jitter / noise.  We expect future systems will be complex enough that it is
hard to estimate the parameters well and it must be understood that the
modeling can only be as good as the input to the models.  Care must be taken to
get good input data from the vendors and the ECP project should place emphasis
on getting quality monitoring data from around DOE datacenters that can seed
modeling efforts on contemporary systems.

Reliability is a topic area which is sensitive not only amongst the government
but also the vendors providing solutions and, as such, thought should be put
into how we are to assemble information and disseminate for maximal progress of
the project while also protecting the participants.

\paragraph{Solution Strategy}
This subproject's solution leverages data collection and analytics to do what
is termed \emph{data-driven datacenter design}.  That is, using data from
supercomputer field data (real and modeled from prospective / hypothetical
machines) to design better datacenters and supercomputers themselves to better
fit our users' needs.  Further, we provide software evaluation tools to emulate
soft faults so users can inject faults into their code and empirically evaluate
their applications' resiliency and vulnerability to corruption.  In this way,
they can develop mitigation strategies and evaluate the efficacy of those
approaches.  It is important to understand, however, that these approaches are
predicated on good input data that these faults are reasonable given what is
expected for target Exascale computer platforms.  The tools provided by this
subproject can create nearly any fault model, but those faults need to be
configured and driven by some information and while we can provide some initial
suggestions it is theoretically possible that those would be incorrect.

\paragraph{Recent Progress}
The modeling efforts in recent months have focused on modeling memory
reliability for prospective machines using publically available reliability
data.  Additionally, there are versions of this modeling effort which use NDA
data as input.

The FSEFI\cite{pfsefigithub,6877352} fault injector recently demonstrated the
capability to inject faults into parallel programs.  Now, using FSEFI, users
can not only perform thousands of individual fault injection trials at the same
time but these individual trials can be parallel programs themselves.  This
effort brought to light some performance improvements needed with FSEFI as well
as enhancements the team is working on to make the tool suite more
user-friendly.

The machine learning and data analytics portion of this subproject have
recently focused on studying data from Los Alamos National Laboratory's largest
classified machine, Trinity, as well as the laboratory's largest cluster used
for open science, Grizzly.  As is common with this kind of analysis, a
surprising amount of investment is needed in simply curating and ``cleaning''
dirty data for analysis.  Initial experiments on this data has focused
studying job logs from systems combined with system / console logs.

%previous 'recent progress'
%The subproject recently completed a milestone where two supercomputer
%resiliency models were created.  One is considerably more complex than the
%other but includes parameters which we found to be hard to measure on
%contemporary systems.  It provides an example of features which we find
%desirable measurement points on future systems.  The simplified model we
%provided we used on the Trinity supercomputer at LANL, the largest system at
%LANL.  Due to the nature of the work performed on the system and the NDA-nature
%of the results of the analysis, we were unable to share publicly the results
%but did share the results in meetings with several laboratories and other
%government agencies, demonstrating the model and showed that it closely matches
%certain parameters from RFP (NDA) documents and neutron beam studies that have
%been published by the team.  

\paragraph{Next Steps}
FY19's focus in this subproject is mostly on stabilization of FSEFI for users
and enhancements of FSEFI by integrating with the
DisCVar\cite{DBLP:conf-ppopp-MenonM18} tool.  DisCVar is a tool that analyzes
static sequential programs and identifies variables that are deemed to have the
most impact of final program correctness.  As such, it can inform developers
requires of an application that need more protections and checks to detect
silent data corruption.  The FSEFI team has already explored DisCVar and is
working with LLNL to connect the tool to FSEFI to inform the fault injector
where to target injections.  It would not be unreasonable to think of this as
multi-scale fault injection where the goal is to use DisCVar to draw attention
to where injections should be performed and then to use FSEFI for intense fault
injection experiments.
% previous 'next steps'
%The FSEFI\cite{pfsefigithub,6877352} fault injector continues development toward a late year milestone of
%parallel fault injection demonstration.  Additionally, there is an upcoming
%milestone for machine learning using data from supercomputer field data.

\subsubsection{LANL ATDM Software Ecosystem \& Delivery Projects - BEE/Charliecloud Subproject} 

\paragraph{Overview}
The BEE/Charliecloud subproject is creating software tools to increase portability
and reproducibility of scientific applications on high performance and cloud
computing platforms.  Charliecloud \cite{priedhorskyrrandlestc2016} is an unprivileged Linux container
runtime.  It allows developers to use the industry-standard Docker
\cite{dockerinc}
toolchain to containerize scientific applications and then execute them on
unmodified DOE facility computing resources without paying any performance
penalty.  BEE \cite{beeproject} (Build and Execution Environment) is a toolkit providing
users with the ability to execute application workflows across a diverse set of
hardware and runtime environments.  Using Bee's tools, users can build and
launch applications on HPC clusters and public and private clouds, in
containers or in containers inside of virtual machines, using a variety of
container runtimes such as Charliecloud and Docker. 

\paragraph{Key Challenges}
Other HPC-focused container runtimes exist, such as NERSC's Shifter
\cite{canonrsjacobsend} and
Singularity \cite{kurtzergmsochatvbauermw}.  These alternative runtimes have characteristics, such as
complex setup requirements and privileged user actions, that are undesirable in
many environments.  Nevertheless, they represent a sizable fraction of the
existing HPC container runtime mindshare.  A key challenge for BEE is
maintaining support for multiple runtimes and the various options that they require
for execution.  This is especially true in the case of Singularity, which
evolves rapidly.  Similarly, there is a diverse collection of resources that
BEE and Charliecloud must support to serve the ECP audience.  From multiple HPC
hardware architectures and HPC accelerators such as GPUs and FPGAs, to
differing HPC runtime environments and resource managers, to a multitude of
public and private cloud providers, there is a large set of available resources
that BEE and Charliecloud must take into consideration to provide a
comprehensive solution.

\paragraph{Solution Strategy}
The BEE/Charliecloud project is focusing first on providing support for
containerized production LANL scientific applications across all of the
existing LANL production HPC systems.  The BEE/Charliecloud components required
for production use at LANL will be documented, released and fully supported.
Follow-on development will focus on expanding support to additional DOE
platforms.  This will mean supporting multiple hardware architectures,
operating systems, resource managers, and storage subsystems.  Support for
alternative container runtimes, such as Docker, Shifter, and Singularity is
planned.

\paragraph{Recent Progress}
% previous recent progress
%Recent Charliecloud progress has been focusing on understanding and documenting
%best practices for running large scale MPI jobs using containerized runtimes.
%Additional work has been done to enhance support for using containers with
%GPUs.  Charliecloud is available at https://github.com/hpc/charliecloud and was
%recently approved for inclusion in the next release of OpenHPC.
%
%BEE currently has beta-level support for launching Charliecloud containers on
%LANL HPC systems.  Automated BEE scalability testing is nearing production
%readiness.
Recent Charliecloud progress has focused on understanding and documenting best
practices for running large scale MPI jobs using containerized runtimes.
Charliecloud is enhancing support for multiple MPI implementations.
Charliecloud is available at https://github.com/hpc/charliecloud and is
distributed inside of Debian and Gentoo Linux distributions as well as being
part of OpenHPC.  Charliecloud won an 2018 R\&D-100 award.

BEE fully supports launching Charliecloud containers on all LANL HPC systems.
It can also launch containers on AWS and OpenStack clouds such as NSF
Chameleon.  BEE also supports interactive launching of jobs with the SLURM
resource manager.

\paragraph{Next Steps}
% previous next steps
%There is an Integrated Project Team (IPT) working at LANL on demonstrating a
%complete BEE/Charliecloud workflow using production applications on production
%HPC systems.  This work will produce a well-documented and packaged BEE release
%consisting of the BEE-Charliecloud launcher, BEEflow workflow management
%system, and support for the Slurm resource manager.
There is an ASC 2019 L2 milestone for LAP, BEE, and Charliecloud in production
and this is the main focus for FY19.  The BEE team requested Q2 FY19 allocation
on Summit (ORNL) to support expanding resource manager support to LSF.

\subsubsection{\stid{6.01} LANL ATDM Tools}

\paragraph{Overview}
The Kitsune Project, part of LANL's ATDM CSE efforts, provides a
compiler-focused infrastructure for improving various aspects of the
exascale programming environment.  At present, our efforts are primarily
focused on advanced LLVM compiler and tool infrastructure to support the
use of a \emph{parallel-aware} intermediate representation.  In
addition, we are actively involved in the Flang Fortran compiler that
is now an official sub-project within the overall LLVM infrastructure.
All these efforts include interactions across ECP as well as
with the broader LLVM community and industry.  

\paragraph{Key Challenges}
A key challenge to our efforts is reaching agreement within the broader community that
a parallel intermediate representation is beneficial and needed within
LLVM.  This not only requires showing the benefits but also providing a
full implementation for evaluation and feedback from the community.
In addition, significant new compiler capabilities represent a
considerable effort to implement and involve many complexities and technical
challenges.  These efforts and the process of up-streaming potential
design and implementation changes do involve some amount of time and
associated risk. 

Additional challenges come from a range of complex issues surrounding
target architectures for exascale systems.  Our use of the LLVM
infrastructure helps reduce many challenges here since many processor
vendors and system providers now leverage and use LLVM for their
commercial compilers.

\paragraph{Solution Strategy}
Given the project challenges, our approach takes aspects of
today's node-level programming systems (e.g. OpenMP and Kokkos) and
popular programming languages (e.g. C++ and Fortran) into
consideration and aims to improve and expand upon their capabilities
to address the needs of ECP.  This allows us to attempt to strike a
balance between incremental improvements to existing infrastructure
and more aggressive techniques that seek to provide innovative
solutions, thereby managing risk while also providing the ability to introduce new
breakthrough technologies.

Unlike current designs, our approach introduces the notion of explicit
parallel constructs into the LLVM intermediate representation, building
off of work done at MIT on Tapir~\cite{2.3.6.01:kitsune:Schardl:2017}. 
We are exploring extensions to this work as well as making some changes
to fundamental data structures within the LLVM infrastructure to assist with
and improve analysis and optimization passes. 

\paragraph{Recent Progress}

Our primary focus is the delivery of capabilities for LANL's ATDM
Ristra application (AD 2.2.5.01).  In support of the requirements for
Ristra, we are targeting the lowering of Kokkos constructs directly
into the parallel-IR representation.  At present, this requires
explicit lowering of Kokkos constructs and we have basic support
for \texttt{parallel\_for} and \texttt{parallel\_reduce} in place.  In
addition we are looking at replacement of LLVM's dominator tree, a key
data structure for optimizations including parallelization and
memory usage analysis, with a \emph{dominator directed-acyclic-graph}
(DAG).  This work is currently underway and should near its initial
implementation early in the 2020 calendar year.  In addition, we are
actively watching recent events within the LLVM community around
multi-level intermediate representations (MLIR) and the relationship
they have with parallel semantics, analysis, optimization, and code
generation.  Furthermore, we are participating in ongoing discussions within the
community about general details behind parallel intermediate forms. 

\paragraph{Next Steps}

The key next step for our feature set is to complete implementation of Kokkos use cases 
that match the needs of Ristra.  Where possible, we will also
explore the broader set of use cases within ECP's overall use of
Kokkos constructs.  The Kitsune toolchain is still very much an
active \emph{proof-of-concept} effort based on Clang (C and C++) with
plans to add support for Fortran via the newly established Flang front
end within LLVM (ST 2.3.2.12).  Even though it is not yet production
ready, we are actively updating and releasing source code and the supporting
infrastructure for deployment as an early evaluation candidate.  In
addition to these components, we will actively begin to explore targeting
the exascale systems (Aurora, Frontier, and El Capitan).


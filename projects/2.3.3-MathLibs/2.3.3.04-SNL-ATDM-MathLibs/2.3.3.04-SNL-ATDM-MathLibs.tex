\subsubsection{\stid{3.04} ATDM SNL Math Libraries} \label{subsubsect:trilinos}

\paragraph{Overview} 
The major project outcome for the SNL ATM Math Libraries project is to provide re-usable and convenient, math-related tools for component-based software engineering of Exascale apps.

This project develops and integrates scalable, modular, and cross-cutting software infrastructure components for ATDM and other future Exascale applications that utilize, where appropriate, ATDM Core CS components: Kokkos performance portability, Sacado/ROL embedded sensitivity analysis and optimization technology, DARMA asynchronous many-tasking, and DataWarehouse.  These components include:
\begin{enumerate}
\item	KokkosKernels (KK):  performance-portable sparse/dense linear algebra and graph kernels that utilize the hierarchical memory subsystem expected in future architectures;
\item	Scalable Solvers (SS):  optimal linear solver algorithms that exploit fine-grain parallelism for vector/SIMT and thread scaling and leverage next-generation execution and communication capabilities;
\item	Agile Components (AC):  tools for interface abstractions, discretization, time integration, and solution of nonlinear PDEs; and 
\item Embedded Analysis (EA):  tools for enabling advanced analysis workflows, focusing on embedded sensitivity analysis and optimization with use of derivatives for uncertainty quantification.
\end{enumerate}

This project combines algorithmic R\&D with delivery of interoperable software components that are expected to be crucial capabilities that will enable Sandia's ATDM application codes to be performance portable across next-generation computing architectures such as GPUs and Xeon Phis.  This work will include integration of these components into the application codes and improving their design and interfaces for mission relevant use cases.	


\paragraph{Key  Challenges}
There are several challenges associated with the work this project is conducting. Part of the complexity arises because profiling tools are not yet full mature for advanced architectures and in this context profiling involves the interplay of several factors which require expert judgment to improve performance.  Another challenging aspect is working on milestones that span a variety of projects and code bases. There is a strong dependence on the various application code development teams for our own team's success. In addition, we face a constant tension between the need for production ready tools and components in a realm where the state-of-the-art is still evolving.

\paragraph{Solution Strategy}
To address the challenges above, the SNL ATM Math Libraries project is taking a staged approach to profiling in regards to target architectures and the algorithms involved. We are also coordinating on a regular basis with the other projects that are involved in our work to minimize impediments. In response to the need for production ready tools, we are focusing on a hierarchical approach that involves producing robust, hardened code for core algorithms while simultaneous pursuing research ideas where appropriate.

\paragraph{Recent Progress}
In this section we provide several high-level snapshots of recent progress in the Math Libraries project
\begin{itemize}
\item The EA team has completed the integration of sensitivity analysis and ROL optimization tools with Tempus in support of transient full-space optimization.
\item The KK team completed development of a prototype for unified SIMD types as Kokkos SIMD type
\item The KK team built batched BLAS kernels based on the Kokkos SIMD type to be run on GPUs. On a P100 GPU, the Kokkoskernels triangular solve and dense matrix-matrix multiply provide up to 2x and 170x speedup, respectively, compared to NVIDIA?s cuBLAS.
\item Profiling of the miniEM application in Panzer by the SS team, using a mesh provided by EMPIRE developers, has identified memory growth in problems during setup which has led to a reduced memory footprint.
\item Initial results, gathered by the SS team, for performance of Tpetra stack (Belos,MueLu,Ifpack2) have been gathered for the solve phase on GPUs. This resulted in the identification of performance bottlenecks that either are in the process of or have been resolved.
\item The AC team completed construction and testing of the operator-split stepper in Tempus. This is required for time integrating the PIC and E-field solves.
\item The AC team implemented and verified a variable time-stepping algorithm (Denner, 2014) within Tempus for use with the BDF2 time-integration scheme and other Tempus schemes. 
\end{itemize}

\paragraph{Next Steps}
The following list details the next steps being taking by each component of the ATDM SNL Math Libraries project:

\begin{itemize}
\item The SS team is investigating, in collaboration with the University of Wyoming, the use of Multilevel solvers, which will eventually get integrated into SPARC via MueLu.
\item The KK team will continue to work on Integration of Kokkoskernels into IFPACK2 line preconditioners as well as testing on Volta GPUs.
\item The AC team will be focusing on the integration of Tempus into EMPIRE-PIC.
\item The EA team will extend the work that was done for adjoint sensitivities to transient, full-space optimization.

\end{itemize}
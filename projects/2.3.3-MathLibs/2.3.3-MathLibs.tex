\subsection{\stid{3} \mathlibs}

\textbf{End State:} Mathematical libraries that (i) interoperate with the ECP software stack; (ii) are incorporated into the ECP applications; and (iii) provide scalable, resilient numerical algorithms that facilitate efficient simulations on Exascale computers.

\subsubsection{Scope and Requirements}
Software libraries are powerful means of sharing verified, optimized algorithms and their implementations. Applied research, development, and support are needed to extend existing DOE mathematical software libraries to make better use of Exascale architectural features. DOE-supported libraries encapsulate the latest results from mathematics and computer science R\&D; many DOE mission-critical applications rely on these numerical libraries and frameworks to incorporate the most advanced technologies available. 

The Mathematical Libraries effort will ensure the healthy functionality of the numerical software libraries on which the ECP applications will depend. The DOE mathematical software libraries used by computational science and engineering applications span the range from light-weight collections of subroutines with simple APIs to more “end-to-end” integrated environments and provide access to a wide range of algorithms for complex problems.

Advances in mathematical and scientific libraries will be necessary to enable computational science on Exascale systems. Exascale computing promises not only to provide more computational resources enabling higher-fidelity simulations and more demanding studies but also to enable the community to pose new scientific questions. Exascale architectural characteristics introduce new features that algorithms and their implementations will need to address in order to be scalable, efficient, and robust. As a result, it will be necessary to conduct research and development to rethink, reformulate, and develop existing and new methods and deploy them in libraries that can be used by applications to deliver more complete and sophisticated models and provide enhanced predictive simulation and analysis capabilities.

The Mathematical Libraries effort must (1) collaborate closely with the Application Development effort (WBS 2.2) to be responsive to the needs of the applications and (2) collaborate with the other products within the Software Technology effort (WBS 2.3) in order to incorporate new technologies and to provide requirements. All software developed within the Mathematical Libraries effort must conform to best practices in software engineering, which will be formulated early in the project in collaboration with the Applications Development focus area. Software produced by this effort must provide scalable numerical algorithms that enable the application efforts to reach their performance goals, encapsulated in libraries whose data structures and routines can be used to build application software.

\subsubsection{Assumptions and Feasibility}
Years of DOE investment have led to a diverse and complementary collection of mathematical software, including AMReX, Chombo, hypre, Dakota, DTK, MAGMA, MFEM, Mesquite, MOAB, PETSc/TAO, PLASMA, ScaLAPACK, SUNDIALS, SuperLU, and Trilinos. This effort is evolving a subset of existing libraries to be performant on Exascale architectures. In addition, research and development is needed into new algorithms whose benefits may be seen only at the extreme scale. Results of preliminary R\&D projects indicate that this approach is feasible.

Additionally, ECP will need to rely on a strong, diverse, and persistent base math research program, which is assumed to continue being supported by the DOE-SC ASCR Office. The ECP technical directors will schedule quarterly meetings with the ASCR research program managers to get updates on research results that might meet ECP requirements as well as to inform the program managers of ECP needs in applications and software components.

\subsubsection{Objectives}
The high-level objective of the Mathematical Libraries effort is to provide scalable, resilient numerical algorithms that facilitate efficient application simulations on Exascale computers. To the greatest extent possible, this objective should be accomplished by preserving the existing capabilities in mathematical software while evolving the implementations to run effectively on the Exascale systems and adding new capabilities that may be needed by Exascale applications.

The key performance metrics for the software developed by this effort are scalability, efficiency, and resilience. As a result of the new capabilities in mathematics libraries developed under this effort, applications will tackle problems that were previously intractable and will model phenomena in physical regimes that were previously unreachable.

\subsubsection{Plan}
As detailed below, the Mathematical Libraries effort supports complementary projects as needed to meet the needs of ECP applications. To ensure relevance to DOE missions, these efforts are DOE-laboratory-led, including strong collaborations with academia, industry, and other organizations, and leveraging existing libraries that are widely used by the DOE HPC community. 

Initial efforts focus on identifying core capabilities needed by selected ECP applications, establishing performance baselines of existing implementations on available Petascale and prototype systems, and beginning re-implementation of lower-level capabilities of the libraries and frameworks. Another key activity is collaborating across all projects in the Mathematical Libraries effort to define community policies in order to enable compatibility among complementary software and to provide a foundation for future work on deeper levels of interoperability. Refactoring of higher-level capabilities will be prioritized based on needs of the applications. In time, these efforts will provide demonstrations of parallel performance of algorithms from the mathematical software on pre-Exascale, leadership-class machines (at first on test problems, but eventually in actual applications). The initial efforts will also inform research into advanced exascale-specific numerical algorithms that will be implemented within the libraries and frameworks. The projects will implement their software on the CORAL, NERSC and ACES systems, and ultimately on initial Exascale systems, so that functionality, performance, and robustness can be evaluated by the applications teams and other elements of the software stack. Throughout the effort the applications teams and other elements of the software stack will evaluate and provide feedback on their functionality, performance, and robustness. These goals will be evaluated at least yearly based on milestones as well as joint milestone activities shared across the associated software stack activities by Application Development and Hardware and Integration project focus areas.

\subsubsection{Risks and Mitigations Strategies}
There are a number of foreseeable risks associated with the Mathematical Libraries effort.
\begin{itemize}
	\item Efficient implementation of new or refactored algorithms to meet Exascale computing requirements may introduce unanticipated requirements on programming environments. To mitigate this risk, effective communication is needed between projects in the Mathematical Libraries effort and projects tasked with developing the programming environments. From the application perspective, this is specifically tracked in a specific AD risk the risk register. Additionally, the risks of an inadequate programming environment overall are tracked as a specific ST risk in the risk register.
	\item A significant number of existing algorithms currently implemented in numerical libraries may scale poorly, thereby requiring significantly more effort than refactoring. The R\&D planned for the first three years of the ECP is the first mitigation for this risk (as well as the co-design centers planned in Application Development). In addition, the ECP will be able to draw from a strong, diverse, well-run, persistent base math research program. From the application perspective, this is tracked via an AD risk in the risk register. Scaling issues for the software stack in general, including libraries, are monitored via an ST risk in the risk register.
	\item Exascale architecture characteristics may force a much tighter coupling among the models, discretizations, and solvers employed, causing general-purpose solvers to be too inefficient. The mitigation strategy is to ensure close collaboration with the sub-elements of the Application Development focus area (WBS 2.2) to understand integration and coupling issues. Again, a strong, diverse, well-run, persistent base math research program may provide risk mitigation strategies.
\end{itemize}

\subsubsection{Future Trends}

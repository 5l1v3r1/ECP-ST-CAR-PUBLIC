\subsubsection{\stid{2.04} SNL ATDM Tools} 

\paragraph{Overview}

The SNL ATDM Tools project is broken into two subprojects: the SNL ATDM DevOps subproject, and the Sandia ATDM Performance Analysis subproject.

The SNL ATDM DevOps subproject focus is on tools and processes supporting development operations (``DevOps'') for the ATDM software development efforts.
DevOps in the SNL ATDM context is all of the software infrastructure development, testing support, integration, and deployment work in support of the ATDM software application and component development teams.

The primary activities of this subproject are to (1) coordinate and prioritize tasks for the various teams that provide DevOps support for ATDM codes, applications, and customers, (2) develop and help deploy shared build, test, and install infrastructure across the ATDM codes and projects, (3) define and support development, testing, integration, and other related workflows for ATDM projects.

The SNL ATDM Performance Analysis subproject is scoped with providing a broad cross-section of portability and performance-related support activities for the laboratories ATDM efforts.
These activities include: (1) providing support for high-performance, hardware-optimized cross-platform builds, including the generation of correct hardware compiler options/software defines; (2) performance
analysis of benchmarking runs, including thread and node scaling on relevant NNSA/ASC testbeds and platforms, and (3) provision for algorithm/code modification or editing of run scripts to optimize performance where issues are identified.

The SNL ATDM Performance Analysis subproject also develops profiling and correctness tools which work with the Kokkos Profiling hooks API.
These tools have been developed to provide insight into the timing of kernels written using Kokkos, as well as data structures utilizing Kokkos parallel containers or Views.
In a number of cases, the profiling tools act as {\em connectors}, establishing a link between important Kokkos performance events and vendor provided tools such as Intel's VTune, NVIDIA's NSight and Arm's MAP profilers.

\paragraph{Key Challenges}

The key challenges associated with this project are the extremely aggressive porting and optimization requirements associated with Sandia's ATDM efforts.
These activities are attempting to port and help support a minimum of three production-class NNSA applications, as well as multiple mini-applications and research prototypes to several ASC-relevant platforms. The first-of-a-kind algorithms being used on these platforms produce complex interactions in the applications that must be fully studied and analyzed to ensure a high level of performance is being offered to the Sandia's user base.

Combined with the application development effort, Sandia is investing heavily in the development of the Trilinos scalable solver stack (used by several codes in ECP and the broader HPC community). The Performance Analysis activity within ATDM is also providing low-level kernel and runtime optimization insight to developers in the Kokkos and Trilinos projects. The DevOps activity within SNL ATDM is providing configuration, build, testing, and workflows tools and processes to keep this stack of software working on the variety of platforms and configurations.

\paragraph{Solution Strategy}

The SNL ATDM Tools has the following primary thrusts:

\begin{enumerate}

\item \textbf{Common Build, Test, and Integration Tools} ensure scalable DevOps efforts and support.

\item \textbf{Testing and Integration Workflows} ensure smooth and productive development and deployment efforts for ATDM software on target platforms.

\item \textbf{High-Performance Applications} ensure well optimized application, library and kernel performance across ASC-relevant computing architectures.

\item \textbf{Performance Portability} ensures performance portability of Sandia ATDM codes across diverse ASC-relevant computing architectures.

\item \textbf{Lightweight Performance Tool Infrastructure} ensures that lightweight tools exist for rapid performance analysis or performance issue identification.

\end{enumerate}

\noindent
The Sandia Performance Analysis sub project was formed from the older Performance Modeling and Analysis Team in 2015.
It's scope was refined to focus specifically on supporting application development activities at the laboratories, with the intent to help provide much stronger levels of performance across the Sandia software portfolio.
The project has provided significant application support since 2015 on topics including application porting and scaling on the ASC Trinity platform, porting to the ASC CTS-1 commodity clusters and has most recently been providing support for the ASC Sierra platform operated by LLNL.

The Kokkos Profiling tools collection was formed in 2015 resulting from research efforts in several successful LDRD projects.
The experimental interface to Kokkos was prototyped in 2014/5 and has since been the default configuration when compiling the Kokkos library.

\paragraph{Recent Progress}

For FY18, the SNL ATDM Tools project has completed the following activities:

\begin{itemize}
\item {\bf Development of Continuous Integration Testing for ASC Platforms} - the DevOps subproject has established a robust, continuous integration suite of tests which are performed over ATS1 (Trinity, Haswell and KNL), ATS2 (CORAL, POWER9/Volta GPU) and ASC CTS1 (Broadwell) systems. These are required platforms for the ATDM code bases. In addition, the subproject has made significant progress on continuous integration pre-merge testing on all of these platforms which will significantly improve code base robustness.

\item {\bf Identification of Testing Performance Issues} - collaboration between the DevOps and tools subprojects has identified an area of potential concern - slow parallel test performance. These occur when the testing infrastructure (CTest) does not have hardware topology/performance awareness and schedules too many tests to use the same resources. Several potential fixes have been investigated and a set of solutions are in prototype form. During FY19, the SNL ATDM project will work to integrate these solutions back into the the CTest testing infrastructure.

\item {\bf Performance Analysis Support} - the performance subproject has continued to provide performance analysis capabilities to ATDM customers, migrating some of its analysis capabilities to include ATS2 (CORAL) and Arm (Vanguard/Astra) platforms in preparation for significant code runs on these machines.
\end{itemize}

\paragraph{Next Steps}

Our next efforts are:

\begin{enumerate}

\item {\bf Evaluation of Spack for Third Party Dependencies} - the DevOps project will work to evaluate Spack as a mechanism for Trilinos third party library build/support.

\item \textbf{Transition ATDM APPs to use unified ATDM Trilinos configurations} - the DevOps project is already well underway with developing unified confgurations for SPARC and EMPIRE which will reduce duplicated work and make more efficient use of build systems.

\item \textbf{Additional Benchmarking:} additional platforms and more extensive benchmarking activities are currently underway, particularly on CORAL POWER9/Volta and Arm-based systems. These studies will have improve the ``day-one'' performance of Sandia's application portfolio on the pre-Exascale ATS platforms when they are released to users for production campaigns in 2019.

\end{enumerate}

\subsubsection{LANL ATDM Tools}

\paragraph{Overview}
The Los Alamos National Laboratory ATDM Tools project provides tools and software 
infrastructure for improving various aspects of the Exascale programming 
environment.  At present our efforts are focused in two key areas that are prioritized 
to meet the needs of LANL's ATDM efforts and are also broadly applicable to broader 
ECP scope: 

\begin{enumerate}

 \item \textbf{Kitsune}: An advanced LLVM~\cite{LLVM:2018} compiler 
        and tool infrastructure supporting a new explicit parallel intermediate 
        representation for the C/C++ and Fortran programming languages. 
        
 \item \textbf{QUO}: A runtime library that assists application developers in the 
        composition of multiple software components (e.g. libraries, multi-physics 
        components) that have disjoint mappings to the underlying hardware components.
        Explicit coordination of this mapping is often critical to not only improve
        performance but also to make more effective use of computing resources (i.e. 
        allowing developers to avoid over-subscription of a job's node allocations).
        
\end{enumerate}

\paragraph{Key Challenges}
The general challenge across both project components is providing and supporting a 
productive development environment in a currently \emph{yet-to-be-defined} set of 
system architectures.  This is not only difficult given a large number of applications 
(and application components) but can become increasingly complex if the attributes of 
potential target platforms are divergent. This requires that we must be flexible and also
understand that many platform-centric decisions can have a significant and potentially 
unexpected impact on our current techniques and software infrastructure. 

\paragraph{Solution Strategy}
Given the project challenges outlined above, our approach takes aspects of today's 
programming systems (e.g. MPI and OpenMP) into consideration and aims to improve and 
expand upon their capabilities to address the needs of Exascale computing across a range
of application areas.  This allows us to attempt to strike a balance between incremental
improvements to existing infrastructure along with more aggressive techniques that seeks 
to provide innovative solutions to help both manage risk and the ability to introduce 
new breakthrough technologies.  

In addition, we are working to form close working relationships with the LLVM community 
to help provide an impactful and longer term set of technologies to the broader computing
community.  These activities span both academic and industry collaborations. 

\paragraph{Recent Progress}
For the Kitsune compiler effort, our recent efforts have focused on supporting an 
infrastructure that maps multiple language constructs from Kokkos, FleCSI and OpenMP into 
an common intermediate representation that explicitly captures the parallel operations 
for analysis and optimization (this is a capability that does not exist in the mainline 
LLVM infrastructure).  This parallel representation may then also be targeted to different
runtime systems (not necessarily those of use by the initial programming system).  See
our most recent paper for a discussion of our overall approach~\cite{Stelle:2017}.  In 
addition, this work will be presented at the upcoming EuroLLVM 
workshop~\cite{EuroLLVM:2018}.  We will continue to use such events to provide the larger
LLVM community with updates to our lessons learned and example use cases of this 
technology. 

The QUO infrastructure has been recently deployed and used by LANL's production codes and
is in active daily use.  Initial steps have been taken to integrate some of its 
functionality into the Kokkos programming system and we have also briefed code teams at
LLNL about the use of the library in multi-physics codes.  Our most recent paper on QUO
highlights the impact of the library's use on application performance across a range of
different applications case studies~\cite{Gutierrez:2017}. 

\paragraph{Next Steps}
Both QUO and Kitsune are working towards quarterly milestones and multiple software 
releases throughout the coming year.  At this point in time QUO has reached a production 
ready state and many activities are focused on performance tweaks, bug fixes and small 
additions to the overall capabilities. Kitsune is still very much an active 
proof-of-concept compiler toolchain focused on C and C++ with future plans to add support
for Fortran via the Flang project~\cite{Flang:2018}.  Even though it is not yet production
ready we are actively releasing source code and the supporting infrastructure for 
deployment as an exploratory and early evaluation candidate. 


\subsubsection{\stid{2.12} Flang}\label{subsubsect:flang}

\paragraph{Overview}

The Flang project provides an open source Fortran
\cite{iso-fortran-2004} \cite{iso-fortran-2010} \cite{iso-fortran-2018}
compiler.  The project was recently formally accepted as a component
of the LLVM Compiler Infrastructure (see \url{http://llvm.org})~\cite{llvm:homepage}.
Leveraging LLVM, Flang will provide a cross-platform Fortran solution available to
ECP and the broad, international LLVM community. The goals of the project include
extending support to GPU accelerators and target Exascale systems, and supporting
LLVM-based software and tools R\&D of interest to a large deployed
base of Fortran applications.

LLVM's growing popularity and wide adoption within the
broader HPC community make it an integral part of the modern software ecosystem.
This project provides the foundation for a Fortran solution that will complement and interoperate with the
Clang/LLVM C++ compiler.  We aim to allow Fortran to grow into a
modernized open source form that is stable, has an active footprint
within the LLVM community, and will meet the needs of a broad
scientific computing community.

\paragraph{Key Challenges}
Today there are several commercially-supported Fortran compilers,
typically available on only one or a few platforms.  None of these are
open source.  While the GNU gfortran open source compiler is available
on a wide variety of platforms, the source base is not modern
LLVM-style C++ and the GPL open source license is not compatible with
LLVM, both of which can impact broader community participation and adoption.

The primary challenge of this project is to create a source base with
the maturity, features, and performance of proprietary solutions with
the cross-platform capability of GNU compilers, and which is licensed
and coded in a style that will be embraced by the LLVM community.
Additional challenges come from robustly supporting all Fortran
language features, programming models, and scalability required for effective use on Exascale systems. 

\paragraph{Solution Strategy}

With the adoption of Flang into the LLVM community as an official subproject,
our strategy focuses on development and delivery of a solid, alternative Fortran
compiler for DOE's Exascale platforms.  In addition, we must be good shepherds
within the LLVM community to establish and grow a vibrant community of our own. This is
in the best interest of ECP as well as the long-term success of Fortran in the
LLVM community and the many industry and academic projects that rely upon it. 

Our path to success will rely on significant testing across not only
the various facilities but also across a very broad and diverse set of applications.
Given the relatively early development stage of Flang, this testing will be paramount 
in the delivery of a robust infrastructure to ECP and the broader community. 

\paragraph{Recent Progress}

After several years of effort and support from NNSA, Flang was successfully ``adopted''
by the LLVM community and is currently in the process of making a transition from a
stand-alone git repository to an LLVM hosted project.  This represents a significant
result and the current code is available as ``F18'' via GitHub:

\begin{center}
\url{https://github.com/flang-compiler/f18}
\end{center}

When it officially completes the move to LLVM's repository it will be renamed ``Flang''. 

The current capabilities of ``F18'' include the full Fortran 2018
standard and OpenMP 5.X syntax and semantics.  As part of the
development of the parsing and semantic analysis portions of the
front-end, over five million lines of Fortran code has been
successfully processed. We will continue and expand this level of
testing as we near the completion of the first full (sequential)
compiler early in the 2020 calendar year.  This effort includes the
use of a Fortran-centric intermediate representation (Fortran IR --
``FIR'') that leverages recent activities within Google
on \href{Multi-Level Intermediate Representations}
{https://www.blog.google/technology/ai/mlir-accelerating-ai-open-source-infrastructure/}
(MLIR) for use with the implementation of FIR.

\paragraph{Next Steps}
Our short-term priorities are focused on the completion of the
sequential compiler, the creation of a significant testing
infrastructure, and helping to lead the interactions and overall
discussions within the LLVM community.  Longer term efforts
will shift to support OpenMP 5.X features critical to ECP applications
on the target Exascale platforms.  We are actively exploring finding a
common leverage point between Clang's current OpenMP code base and
Flang.  This would enable the reuse of existing code versus writing
everything from scratch in Flang.  We see this as a critical path
forward to enabling a timely release of a node-level parallelizing
compiler for ECP.  Additional work will focus on features that would
benefit Fortran within the LLVM infrastructure as well as general and
targeted optimization and analysis capabilities.


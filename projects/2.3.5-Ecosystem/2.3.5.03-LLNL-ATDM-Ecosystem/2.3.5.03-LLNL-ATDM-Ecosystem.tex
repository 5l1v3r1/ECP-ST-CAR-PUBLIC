\subsubsection{\stid{5.03} LLNL ATDM SW Ecosystem \& Delivery: DevRAMP sub-project}

\paragraph{Overview} The {\bf DevRAMP} (Reproducibility, Analysis,
Monitoring, and Productivity) project is creating tools and services to
multiply the productivity of HPC developers through several major
software efforts:

\begin{enumerate}
    \item {\bf Spack} is a package manager for
    HPC~\cite{melara+:cug17,gamblin+:sc18-spack-tutorial,gamblin+:sc18-spack-bof,gamblin+:sc17-spack-tutorial,gamblin+:sc16-spack-tutorial,gamblin+:sc15,gamblin+:llnl-spack-tutorial-17,gamblin+:ecp18-spack-tutorial,gamblin:pp18-spack,gamblin:hpckp17,gamblin:fosdem18-compilers,gamblin:fosdem18-binary,gamblin:ecp18-spack-sotu,gamblin:eb-user-meeting18,becker+:hust16}.
    It automates the process of downloading, building, and installing
    different versions of HPC applications, libraries, and their
    dependencies.  Facilities can manage multi-user software deployments,
    and developers and users can manage their own stacks separately.
    Spack enables complex applications to be assembled from components,
    lowers barriers to reuse, and allows builds to be reproduced easily.

    \item {\bf Sonar} is a data cluster and a software stack for
    monitoring and analysis. It allows developers and facility staff to
    understand how supercomputers and applications interact. Sonar
    aggregates data from system-level monitoring services,
    application-level measurements, and facility power/temperature
    meters. Users analyze monitored data in Jupyter notebooks. Our focus
    is on security for the services running on Sonar (e.g., JupyterHub),
    and on developing the sophisticated {\it ScrubJay} query tool.
\end{enumerate}

\paragraph{Key Challenges}
{\it Spack} makes HPC software complexity manageable. Obtaining optimal
performance on supercomputers is a difficult task; the space of possible
ways to build software is combinatorial in size, and software reuse is
hindered by the complexity of integrating a large number of packages and
by issues such as binary compatibility.  Spack makes it easy to build
optimized, reproducible, and reusable HPC software.

{\it Sonar} aims to find the root causes of performance variability of
jobs that run at Livermore Computing (LC).  A single application's runtime may
depend on the system it runs on, the file system it uses, other jobs
running at the same time, etc. Sonar aims to provide an integrated view
of application performance that allows developers to understand the full
range of performance factors affecting their applications.

\paragraph{Solution Strategy}
{\it Spack} provides a domain-specific language for templated build
recipes.  It provides a unique infrastructure called the {\it
concretizer}, which solves the complex constraint problems that arise in
HPC dependency resolution.  Developers can specify builds {\it
abstractly}, and Spack automates the tedious configuration process and
drives the build. Spack also includes online services to host recipes,
code, and binaries for broad reuse.  These repositories are maintained by
Spack's very active community of contributors.


{\it Sonar} is a data cluster and a suite of monitoring tools and
front-end analysis services. The Sonar project focuses on adding security
to components of this stack so that they can be used in multi-tenant HPC
environments like LC. Sonar has secured and deployed Cassandra, Spark,
Kafka, and it has developed a parallel data analysis tool called
ScrubJay. Notably, the Sonar project contributes features to upstream
open source projects like JupyterHub, so that they can be run securely
not only at LC, but at any HPC center.

\paragraph{Recent Progress}
In FY18, the Sonar team secured Apache Kafka and expanded continuous
production monitoring to include three major LC production clusters. The
team completed work on securing the open source JupyterHub
tool~\cite{mendoza:jupyterhub-ssl}. For the first time, any user at an
HPC center can launch an analysis notebooks securely on bare-metal
clusters. This functionality is integrated and available to any user of
the tool.

The Spack team completed two major projects: virtual {\it environments}
that improve reproducibility and enhance developer workflow, and a cloud
{\it build service} implemented in GitLab. Optimized Spack binaries are
now precompiled so that in many cases Spack users will not need to build
their own software.  The build service greatly accelerates deployment
times, and it enables Spack releases to be rigorously tested.

\paragraph{Next Steps}
\begin{enumerate}

    \item {\bf Expanded analytics for Sonar.} The sonar team will use the
    recently deployed JupyterHub system to develop detailed analyses of
    facility data that provide insight into application performance.

    \item {\bf Spack Stacks and Concretization.} The Spack team is
    developing {\it spack stacks}, which will allow a facility to
    describe and deploy an entire combinatorial software stack, including
    module files, with a single, concise YAML file. The team is also
    working on concretizer enhancements to better support binary caches
    and build reuse. These enhancements will adapt Spack to use a
    sophisticated SAT solver. Spack will provide a complete software
    deployment pipeline users, developers, and facilities.
\end{enumerate}


\subsubsection{\stid{5.03} LLNL ATDM SW Ecosystem \& Delivery: Flux sub-project}
\paragraph{Overview}
The {\bf Flux} scheduling project aims to identify, define, and build the
common infrastructures and interfaces to support the scheduling of
exascale workflows. Flux is LLNL's next-generation resource and job
management (RJM) system for its clusters, as well as a user-level RJM
platform for exascale systems.


\paragraph{Key Challenges}
RJM systems are central to enabling efficient execution of HPC
applications, and they are users' main interface for scheduling complex
workflows.
%
Emerging million- and billion-way parallel pre-exascale workflows
significantly complicate efficient (co-)scheduling and execution.
%
Traditional \emph{centralized} techniques implemented within RJMs such as
SLURM~\cite{SLURM}, LSF~\cite{LSF}, MOAB~\cite{MOAB}, and PBS
Pro~\cite{PBSPro} are aimed at small numbers of large, homogeneous,
long-running jobs.  Exascale workflows will comprise many, often small
and short-running, heterogeneous elements.
%
Key challenges include:
{\it throughput}:
      large ensemble simulations generate very large numbers
      of jobs that cannot be scheduled efficiently with traditional approaches;
{\it co-scheduling}: complex coupling requires
      sophisticated co-scheduling that existing tools cannot easily provide;
{\it job coordination and communication}:
      existing approaches lack interfaces for fine-grained interactions between
      jobs;
and {\it portability}:
      many ad-hoc user-level schedulers attempt to address these
      challenges, but they are not portable or scalable (e.g., they may use
      millions of small files).

\paragraph{Solution Strategy}
The Flux framework is a suite of tools and libraries that can be used to
build custom resource managers and schedulers at HPC sites.  Flux's
architecture is fully hierarchical, and it allows for seamless nesting in
a highly scalable, customizable, and resilient manner.  This architecture
addresses the key exascale workflow scheduling challenges in a scalable,
easy-to-use, and portable manner. Flux is based on four distinct
scheduling techniques: parallelism for throughput, specialization for
co-scheduling, rich APIs for easy job coordination and communication, and
a consistent API for portability~\cite{FluxSC18}.

\paragraph{Recent Progress}
Together with application teams, the Flux team co-designed workflow
solutions for on LLNL's largest machines, including the pre-exascale
system, Sierra.  Flux enabled two major workflows during Sierra's early
science period: one for the Cancer Moonshot and another for LLNL's
strategic Machine Learning project.  Integrating Flux enabled these teams
to scale to billion-way concurrent ensembles on Sierra, and these teams
now use Flux in production both on Sierra and on Linux clusters.

\paragraph{Next Steps}
\begin{enumerate}

    \item {\bf Job-dependency Management and Exception Model Co-design
    with LLNL UQP team}: The flux teams will work on scalable approaches
    to handle exceptions raised during ensemble workflows. Current
    approaches to this problem are ad-hoc and machine-dependent.  Flux
    will formalize exception types and partner with LLNL's UQP team to
    cooperatively detect, propagate, and handle exceptions.

    \item {\bf DYnamic and Asynchronous Data Streamliner (DYAD)}: Sierra
    early science applications exposed challenges for efficiently
    data-sharing among co-scheduled workflow elements.  The Flux project
    will develop a general-purpose data-transfer infrastructure using Flux
    APIs, so that jobs can easily implement producer-consumer workflows.

    \item {\bf Performance Variability-Aware Scheduling}: Performance
    variation poses a challenge to workflows reproducibility.  The team
    will extend Flux’s graph-based resource matching service to enable
    variation-aware scheduling policies that mitigate variation.

    \item {\bf Steer and Co-design CORAL2 RM APIs}: The team will
    co-design various RJM interfaces, e.g. power monitoring and control,
    in collaboration with one of the CORAL2 vendors, with the ultimate
    goal of using Flux as the CORAL2 RJM system.

\end{enumerate}
